\documentclass[twoside]{article}
    \setlength{\oddsidemargin} {0.25 in}
    \setlength{\evensidemargin}{-0.25 in}
    \setlength{\topmargin}     {-0.6 in}
    \setlength{\textwidth}     {6.5 in}
    \setlength{\textheight}    {8.5 in}
    \setlength{\headsep}       {0.75 in}
    \setlength{\parindent}     {0 in}
    \setlength{\parskip}       {0.1 in}

\usepackage{amsmath}
\usepackage{amsfonts}
\usepackage{graphicx}
\usepackage{amssymb}
\usepackage{mathrsfs}   % provides \mathscr
\usepackage{physics}    % provides \norm

%
% The following commands set up the lecnum (lecture number)
% counter and make various numbering schemes work relative
% to the lecture number.
%
\newcounter{lecnum}
    \renewcommand{\thepage}{\thelecnum-\arabic{page}}
    \renewcommand{\thesection}{\thelecnum.\arabic{section}}
    \renewcommand{\theequation}{\thelecnum.\arabic{equation}}
    \renewcommand{\thefigure}{\thelecnum.\arabic{figure}}
    \renewcommand{\thetable}{\thelecnum.\arabic{table}}

%
% The following macro is used to generate the header.
%
\newcommand{\lecture}[4]{
    \pagestyle{myheadings}
    \thispagestyle{plain}
    \newpage
    \setcounter{lecnum}{#1}
    \setcounter{page}{1}
    \noindent
    \begin{center}
        \framebox{
            \vbox{
                \vspace{2mm}
                \hbox to 6.28in {{\bf CS 217: Artificial Intelligence and
                                  Machine Learning \hfill Jan 10, 2024}}
                \vspace{4mm}
                \hbox to 6.28in {{\Large \hfill Lecture #1: #2  \hfill}}
                \vspace{2mm}
                \hbox to 6.28in {{\it Lecturer: #3 \hfill Scribes: #4}}
                \vspace{2mm}
            }
        }
    \end{center}
    \markboth{Lecture #1: #2}{Lecture #1: #2}

    \textbf{Disclaimer:} \textit{These notes aggregate content from several
    texts and have not been subjected to the usual scrutiny deserved by
    formal publications. If you find errors, please bring to the notice of
    the Instructor.}

    \vspace*{1mm}
}

%
% Convention for citations is authors' initials followed by the year.
% For example, to cite a paper by Leighton and Maggs you would type
% \cite{LM89}, and to cite a paper by Strassen you would type \cite{S69}.
% (To avoid bibliography problems, for now we redefine the \cite command.)
% Also commands that create a suitable format for the reference list.
%
\renewcommand{\cite}[1]{[#1]}
    \def\beginrefs{\begin{list}%
            {[\arabic{equation}]}{\usecounter{equation}
             \setlength{\leftmargin}{2.0truecm}\setlength{\labelsep}{0.4truecm}%
             \setlength{\labelwidth}{1.6truecm}}}
    \def\endrefs{\end{list}}
    \def\bibentry#1{\item[\hbox{[#1]}]}

% Use this command for a figure; it puts a figure in wherever you want it.
% usage: \fig{NUMBER}{JUST-THE-FILENAME}{CAPTION}
\newcommand{\fig}[3]{
    % \vspace{#2}
    \begin{center}
    \includegraphics{figures/#2} 
    \newline
    Figure \thelecnum.#1:~#3
    \end{center}
}
	
% Use these for theorems, lemmas, proofs, etc.
\newtheorem{theorem}{Theorem}[lecnum]
\newtheorem{lemma}[theorem]{Lemma}
\newtheorem{proposition}[theorem]{Proposition}
\newtheorem{claim}[theorem]{Claim}
\newtheorem{corollary}[theorem]{Corollary}
\newtheorem{definition}[theorem]{Definition}
\newenvironment{proof}{{\bf Proof:}}{\hfill\rule{2mm}{2mm}}

\begin{document}

\lecture{1}{The Basics of Optimization}{Swaprava Nath}{Aditya Singh, Dion Reji,
Shreyas Katdare, Brian Mackwan}

In this lecture, we discuss some basic optimization problems.

\begin{tabular}{l | l | l | l}
    & Variable & Solution space & Solution complexity \\
    \hline
    Continuous optimization & take any real value (continuous) & infinite &
    polynomial in the size of the problem \\
    Discrete optimization & discrete & finite & exponential (e.g., knapsack)
\end{tabular}

\section{What is optimization?}
An optimization problem involves finding the minimum value attained
by a function subject to some constraints, i.e., \[
    \min_{x \in \mathscr{C}} f(x),
\] where $f(x)$ is the objective function and $\mathscr{C}$ is the
constraint set.

\textbf{Example.} Minimize $(x - 2)^2$ with the constraint that
$x \in [0, 1] \cup [4, 7].$

Here, the objective function is $f(x) = (x - 2)^2$ and the constraint set is
$\mathscr{C} = [0, 1] \cup [4, 7].$ This is a one-variable function, and we can
easily see from the plot in figure REF that $x^* = 1$ is the optimal value of
$x$.

% Insert plot of $(x - 2)^2$ to the right, made in TikZ.

\textbf{Example.} We want to optimally place a warehouse, so that the sum of the
Euclidean distances between the warehouse and the cities is minimized.

Let the cities be located at $\mathbf{y}_1, \ldots, \mathbf{y}_m$ and the
warehouse at $\mathbf{x}$. We want to find the following: \[
    \min_{x \in \mathscr{C}} \sum_{i = 1}^{m} \|\mathbf{x} - \mathbf{y}_i\|_2,
\] where $\mathscr{C}$ is the set of all points where we want our Warehouse to
be, and $\| \cdot \|_2$ represents the $L^2$ norm, defined by \[
   \|x\| := \sqrt{x_1^2 + x_2^2},
\] where $\mathbf{x} = \begin{bmatrix} x_1 \\ x_2 \end{bmatrix}$.

% Insert figure

\textbf{Example} (Image de-blurring). We consider grayscale images of size $m
\times n$, where each pixel has an intensity value in $[0, 1].$ The input image
is $\mathbf{y} = [y_{i, j}]^{m \times n}$ and the desired output is
$\mathbf{x}.$
\[
    \min_{\mathbf{x} \in [0, 1]^{m \times n}}
    \left(
    \sum_{i = 0}^{m - 1} \sum_{j = 0}^{n - 1} \|y_{i, j} - (k * \mathbf{x})_{i, j}\| +
    \lambda \sum_{i = 0}^{m - 1} \sum_{j = 0}^{n - 1} ((\mathbf{x}_{i, j} - \mathbf{x}_{i, j + 1})^2 + (\mathbf{x}_{i + 1, j} - \mathbf{x}_{i, j})^2)
    \right)
\] Here, $\lambda$ and $k$ are the hyperparameters, which are determined by
experiment. The optimal value of $\lambda$ and $k$ depend on the image. For
example, medical images may require different $\lambda$ and $k$ as compared to
images of trees.

% Insert figure

\textbf{Example} (Machine learning; curve-fitting). We have inputs $(x_i, y_i),$
where $i \in [n].$ We want to find \[
    \min_\Theta \sum_{i = 1}^{n} \ell(h_\theta(x_i), y_i),
\] where $\ell(\cdot, \cdot)$ is a loss function, $h_\Theta(x) = w_0 + w_1 x +
w_2 x^2$ is the hypothesis (so $h_\theta(x_i)$ is the hypothesized point), and
$\Theta = (w_0, w_1, w_2)$. An example of a loss function is \[
    \ell(y_1, y_2) = (y_1 - y_2)^2.
\] 

\end{document}
